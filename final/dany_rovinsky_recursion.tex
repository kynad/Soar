\documentclass[3p,times]{elsarticle} % final,
\usepackage{times}
\usepackage[T1]{fontenc}
\usepackage[latin1]{inputenc}

\usepackage{verbatim}
\usepackage[active]{srcltx}

\usepackage{draftwatermark}
\SetWatermarkText{\today}
\SetWatermarkScale{0.8}
%\usepackage[onEveryPage]{coffee}
%\usepackage[]{coffee}

% Figures
\usepackage{wrapfig}
% \usepackage{etoolbox}
\usepackage{graphicx}
\usepackage{subfigure}

% Algorithms
\usepackage[ruled]{algorithm}
%\usepackage{algorithm}
%\usepackage[noend]{algorithmic}
\usepackage{algorithmicx}
\usepackage[noend]{algpseudocode}

% Math
\usepackage{amsmath}
\usepackage{amsthm}
\usepackage{amssymb}

\newtheorem{theorem}{Theorem}[section]
\newtheorem{conjecture}{Conjecture}[section]
\newtheorem{corollary}{Corollary}[section]
\newtheorem{definition}{Definition}[section]

% Math notations
\newcommand{\horizon}{\mathop{\mathrm{horizon}}}
\newcommand{\support}{\mathop{\mathrm{support}}}
\newcommand{\pre}{\mathop{\mathrm{preconds}}}
\newcommand{\term}{\mathop{\mathrm{termconds}}}


% Procedure calls
\newcommand{\noun}[1]{\textsc{#1}}

\newcommand{\alloc}{\mathop{\mathcal{ALLOCATE}}}
\newcommand{\vote}{\mathop{\mathcal{VOTE}}}
\newcommand{\sstart}{\mathop{\mathcal{S.START}}}
\newcommand{\sstop}{\mathop{\mathcal{S.STOP}}}


\newcommand{\inform}{\mathop{\mathcal{INFORM}}}
\newcommand{\fuse}{\mathop{\mathcal{FUSE}}}
\newcommand{\receive}{\mathop{\mathcal{RECEIVE}}}


\begin{document}

\begin{frontmatter}
\title{Individual Plan Execution with Recursion}
\author{Dany Rovinsky\\304359276}
\address{Computer Science Department\\
Bar Ilan University, Israel}
\ead{dany.rovinsky@gmail.com}

\begin{keyword}
Distributed multi-robot systems, collaboration, teamwork, planning
\end{keyword}

\begin{abstract}
TODO
\end{abstract}

\end{frontmatter}

\section{Introduction}

The original \textit{BIPE2} algorithm was delegating impasse resolution to a \noun{Choose} procedure that was assumed to be given. In contrast to the original work, the \textit{BIPER} algorithm suggested in this paper assumes an impasse resolution has a form of a plan, rather than a procedure and uses recursive calls to itself in order to execute this plan to solve both sequential and hierarchical impasses.

\begin{algorithm}[htbp]
\caption{Individual decision-making algorithm \textbf{with recursion}.}
\label{alg:ibc}

\begin{algorithmic}[1]
\small

\Require Knowledebase $W$
\Require Impasse resolution plan \textit{Choose}
\Require Impasse set variable name \textbf{IS}
\Require Impasse resolution variable name \textbf{IR}
\Require Condition Testing Procedure \noun{Test}
\Require Belief Update Procedure \noun{Update}
\Require Belief Revision Procedure \noun{Revise}
\Require Start Execution Procedure \noun{Start}
\Require Stop Execution Procedure \noun{Stop}
\Require copy Execution Procedure \noun{copy}

\Statex

\Procedure{BIPER}{Plan $\langle B,H,N,b_{0}\rangle$, Knowledgebase $KB$}

\State $S\gets\emptyset$ \Comment New execution stack
\State $b\gets b_0$
\State \Call{Push}{$S$,$b$} \label{alg:ibc:push}
% Allocation
\While{$\exists n$, where $(b,n)\in H$} \Comment $b$ can be decomposed \label{alg:ibc:allocate}
  \State $A\gets \{n | (b,n)\in H\}$  \Comment children of $b$
  \State $\overline{KB}\gets copy(KB)$ \Comment create a copy of the local knowledgebase.
  \State $\overline{KB}[\textit{IS}]\gets \{a | a\in A, \Call{Test}{\pre(a),W}\}$ \Comment store the impasse set in the new local Knowledgebase.
  \State $\Call{BIPER}{Choose, \overline{KB}$ \label{alg:ibc:choose-allocate}\Comment Choose among $C$ behaviors
  \State \Call{Push}{$S$,$\overline{KB}[IR]$}
\EndWhile \label{alg:ibc:push-end}

\Statex

% Execution
\ForAll{$s\in S$}\label{alg:ibc:exec} \Comment Start execution of all behaviors in $S$
 \State {\bf if $s$ not running} \Call{Start}{$s$} 
\EndFor

\Statex 

\State $E\gets\emptyset$ \label{alg:ibc:monitor-exec}
\While{$E=\emptyset$} \Comment $E$ is the set of terminating behaviors
  \State $K\gets\Call{Update}{W}$
  \State $W\gets\Call{Revise}{W,K}$
  \State $E\gets \{a | a\in S, \Call{Test}{\term(a),W}\}$ \Comment Check termination conditions
\EndWhile  \label{alg:ibc:exec-end}

\Statex

% Stop Execution
\While{$E\neq\emptyset$} \label{alg:ibc:terminate} \Comment Stop and pop all terminating behaviors and descendants
  \State $e\gets Pop(S)$
  \State $\Call{Stop}{e}$
  \If{$e\in E$}
    \State $E\gets E-\{e\}$
  \EndIf
\EndWhile \label{alg:ibc:terminate-end}

\Statex

% Next
\State $A\gets \{n | (e,n)\in N\}$ \Comment sequential followers of $e$, topmost terminating behavior
\State $C\gets \{a | a\in A, \Call{Test}{\pre(a),W}\}$
\If{$C\neq\emptyset$} \Comment There are potential followers
  \State $\overline{KB}\gets copy(KB)$ \Comment create a copy of the local knowledgebase.
  \State $\overline{KB}[\textit{IS}]\gets C$ \Comment store the impasse set in the new local knowledgebase.
  \State $\Call{BIPER}{Choose, \overline{KB}}$ \label{alg:ibc:choose-vote} \Comment Choose among $C$ behaviors
  \State $b\gets \overline{KB}[IR]$ \Comment get the impasse resolution and store in $b$.
  \State {\bf Goto} \ref{alg:ibc:push}
\EndIf
\State $b\gets\Call{Peek}{S}$ \Comment{No potential followers, continue with parent}
\If{$b\neq\emptyset$}
  \State {\bf Goto} \ref{alg:ibc:monitor-exec} % \ref{alg:ibc:allocate}
\EndIf

\EndProcedure
\end{algorithmic}

\end{algorithm}

%TODO: Explain algorithm
Besides the \noun{Choose} procedure, Algorithm~\ref{alg:ibc} also relies on procedures for accessing a stack
which holds the behaviors selected for execution, for getting new updates from the world modeling processes
(\noun{Update}), and for merging these updates with the knowledge-base that the robot maintains of its
beliefs (\noun{Revise}).  The procedures \noun{Start} and \noun{Stop} are used to begin and end
execution of a given behavior. In that, I am adopting a view of behaviors as separate threads,
which are controlled from Algorithm~\ref{alg:ibc}.

\section{Recursion:}

Outline:
\begin{itemize}
  \item Why recursion?
  \item CHOOSE as a plan, not a function (CaaP?). Same assumptions, only now I assume the way to resolve impasses has a form of a plan.
  \item Communication protocol over KB.
  \item The algorithm: Instead of set $C$, the key-value "imapsse-(hierarchical|sequential)-set-\$depth\_counter" : $C$ will be added to KB. Instead of CHOOSE procedure a recursive call to Algorithm\ref{alg:ibc} will be used. Then, the value of "impasse-(hierarchical|sequential)-resolution-\$depth\_counter" from KB will be pushed to the stack. In the sequential case, there might be a need to special handling of the empty set case.
\end{itemize}

\end{document}
