\documentclass[3p,times]{elsarticle} % final,
\usepackage{times}
\usepackage[T1]{fontenc}
\usepackage[latin1]{inputenc}

\usepackage{verbatim}
\usepackage[active]{srcltx}

\usepackage{draftwatermark}
\SetWatermarkText{\today}
\SetWatermarkScale{0.8}
%\usepackage[onEveryPage]{coffee}
%\usepackage[]{coffee}

% Figures
\usepackage{wrapfig}
% \usepackage{etoolbox}
\usepackage{graphicx}
\usepackage{subfigure}

% Algorithms
\usepackage[ruled]{algorithm}
%\usepackage{algorithm}
%\usepackage[noend]{algorithmic}
\usepackage{algorithmicx}
\usepackage[noend]{algpseudocode}

% Math
\usepackage{amsmath}
\usepackage{amsthm}
\usepackage{amssymb}

\newtheorem{theorem}{Theorem}[section]
\newtheorem{conjecture}{Conjecture}[section]
\newtheorem{corollary}{Corollary}[section]
\newtheorem{definition}{Definition}[section]

% Math notations
\newcommand{\horizon}{\mathop{\mathrm{horizon}}}
\newcommand{\support}{\mathop{\mathrm{support}}}
\newcommand{\pre}{\mathop{\mathrm{preconds}}}
\newcommand{\term}{\mathop{\mathrm{termconds}}}


% Procedure calls
\newcommand{\noun}[1]{\textsc{#1}}

\newcommand{\alloc}{\mathop{\mathcal{ALLOCATE}}}
\newcommand{\vote}{\mathop{\mathcal{VOTE}}}
\newcommand{\sstart}{\mathop{\mathcal{S.START}}}
\newcommand{\sstop}{\mathop{\mathcal{S.STOP}}}


\newcommand{\inform}{\mathop{\mathcal{INFORM}}}
\newcommand{\fuse}{\mathop{\mathcal{FUSE}}}
\newcommand{\receive}{\mathop{\mathcal{RECEIVE}}}


\begin{document}

\begin{frontmatter}
\title{Recursive Plan Execution}
\author{Dany Rovinsky}
\address{Computer Science Department\\
Bar Ilan University, Israel}
\ead{dany.rovinsky@gmail.com}

\begin{keyword}
Distributed multi-robot systems, collaboration, teamwork, planning
\end{keyword}

\begin{abstract}
A common approach to building robot control systems
is algorithmic in nature; it presupposes that robots are tools to be manipulated perfectly by
an algorithm built specifically to optimally solve a given task. But the complexities of realistic tasks and missions
in real-world environments are not easily addressed
by this approach.  This paper posits that \emph{robots are agents}, who should be capable of execution-time
reasoning and management of their interactions with their peers. Such reasoning alleviates reliance
on being able to account for every possible interaction in advance.  To manage these interactions
in execution time, this paper presents BIPE2 (Bar Ilan Plan Execution Engine), a general plan-execution 
algorithm. The design of BIPE2 neatly separates knowledge (beliefs) and decision-making procedures (plans).
\end{abstract}

\end{frontmatter}

\begin{algorithm}[htbp]
\caption{Individual decision-making algorithm.}
\label{alg:ibc}
\input{algorithms/individual-no-recursion.tex}
\end{algorithm}

%TODO: Explain algorithm
%- explain going to the parent, do a quick run, ...
Besides the \noun{Choose} procedure, Algorithm~\ref{alg:ibc} also relies on procedures for accessing a stack
which holds the behaviors selected for execution, for getting new updates from the world modeling processes
(\noun{Update}), and for merging these updates with the knowledge-base that the robot maintains of its
beliefs (\noun{Revise}).  The procedures \noun{Start} and \noun{Stop} are used to begin and end
execution of a given behavior. In that, I am adopting a view of behaviors as separate threads,
which are controlled from Algorithm~\ref{alg:ibc}.

\section{Recursion:}

Outline:
\begin{itemize}
  \item Why recursion?
  \item CHOOSE as a plan, not a function (CaaP?). Same assumptions, only now I assume the way to resolve impasses has a form of a plan.
  \item Communication protocol over KB.
  \item The algorithm: Instead of set $C$, the key-value "imapsse-(hierarchical|sequential)-set-\$depth\_counter" : $C$ will be added to KB. Instead of CHOOSE procedure a recursive call to Algorithm\ref{alg:ibc} will be used. Then, the value of "impasse-(hierarchical|sequential)-resolution-\$depth\_counter" from KB will be pushed to the stack. In the sequential case, there might be a need to special handling of the empty set case.
\end{itemize}

\end{document}
